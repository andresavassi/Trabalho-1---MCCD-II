\documentclass{beamer} % Define a classe do documento como Beamer

\usepackage{listings}
\usepackage{xcolor} % necessário para as cores no listings
\usepackage{listings}
\usepackage{xcolor}
\usepackage{graphicx}
\usepackage[percent]{overpic}
\usepackage{fancyvrb}
\usepackage[utf8]{inputenc}
\usepackage{amsmath}
\usepackage{algorithm}
\usepackage{algorithmic}
\usepackage{booktabs}
\usepackage{amsmath}

\lstset{
	language=R,
	basicstyle=\ttfamily\small,
	keywordstyle=\color{blue},
	commentstyle=\color{green!50!black},
	literate={\#}{{\texttt{\#}}}1,  % Processa o # corretamente
	% ... outras configurações
}

\usetheme{Madrid}   
\graphicspath{{figuras/}}  

% Informações do Título
\title{Simulated Annealing \\ (Recozimento Simulado)}
\author{}
\institute{UFMG}
\date{10/11/2025} % Data da apresentação

% CONFIGURAÇÃO PERSONALIZADA DO RODAPÉ
\setbeamertemplate{footline}{
	\leavevmode%
	\hbox{%
		% ESQUERDA: "grupo 2"
		\begin{beamercolorbox}[wd=.33\paperwidth,ht=2.25ex,dp=1ex,left]{author in head/foot}%
			\hspace{0.5cm}\textbf{Grupo II}%
		\end{beamercolorbox}%
		% CENTRO: "MCCD -2"
		\begin{beamercolorbox}[wd=.34\paperwidth,ht=2.25ex,dp=1ex,center]{title in head/foot}%
			\textbf{MCCD - II}%
		\end{beamercolorbox}%
		% DIREITA: Número da página (opcional)
		\begin{beamercolorbox}[wd=.33\paperwidth,ht=2.25ex,dp=1ex,right]{date in head/foot}%
			\textbf{10/11/2025}
			\insertpagenumber\hspace{0.5cm}%
	\end{beamercolorbox}}%
	\vskip0pt%
}

% Início do Documento
\begin{document}
	
	% Slide de Título
	\begin{frame}
		\titlepage % Gera o slide de título com as informações acima
	\end{frame}
	
	% Sumário (Opcional, mas recomendado)
	\begin{frame}{Sumário}
		\tableofcontents % Gera um sumário automático baseado nas seções
	\end{frame}
	\footnotesize
	
	
	\section{Simulação de Recozimento}
	\subsection{Introdução}
	\begin{frame}{Recozimento Simulado - Introdução}
	Também chamado de método meta-heurístico, este tem como intuito resolver problemas de otimização de grande complexidade. Este algoritmo foi introduzido por volta de 1980 por três pesquisadores, sendo eles: Kirkpatrick, Gelatt e Vecchi. O método traz soluções sub-ótimas, ou seja, eficientes e sem grande esforço computacional.
	
	Tem como objetivo encontrar resultados satisfatórios, ainda que não exatos, sendo que muitas vezes estes resultados são inviáveis e/ou difíceis de serem alcançados.
	
	Se enquadra na classe \textit{Markov chain Monte Carlo} (MCMC) e, por ter caráter estocástico alinhado à sua capacidade de escapar de mínimos locais, este é um ótimo candidato para resolução de problemas complexos de otimização.
	
	
	\end{frame}
	
	\subsection{Processo de Recozimento}
	
	\begin{frame}{Processo de Recozimento}
	O conceito foi introduzido da ideia de recozimento, onde um sólido é levado para um estado de baixa energia após aumentar sua temperatura.
	
	O processo consiste em dua estapas, sendo elas:
	
	\begin{itemize}
	\item "Derretimento" da sua estrutura ao levá-lo a uma temperatura muito elevada.
	\item Resfriamento esquematizado buscando atingir um estado sólido de energia mínima.
	\end{itemize}
	Uma vez em estado líquido as partículas do material exposto são distribuídas de forma aleatória.
	
	O estado mínimo de energia só é alcançado com uma temperatura inicial que seja suficientemente alta e um tempo de resfriamento que seja logo o suficiente.
	\end{frame}
	
	
	\subsection{Aplicabilidade}
	\begin{frame}{Aplicabilidade}
	 O algoritmo \textbf{Simulated Annealing} tem aplicabilidade em várias áreas e aqui serão expostas algumas delas:
	 \begin{itemize}
	 \item Engenharia de software
	 \item \textit{Machine Learning}
	 \item Teoria de filas
	 \item Processos de manufatura
	 \item logística e transporte com otimização
	 \end{itemize}
	\end{frame}
	
	
	\subsection{Teoria}
	\begin{frame}{Dist. Weibull Tri-Paramétrica}
	O estudo será centrado em estimar os parâmetros primeiramente de uma distribuição Weibull com a seguinte função densidade:
	
	\begin{equation}
	 f(x) = \frac{\beta}{\eta}\left(\frac{x- \gamma}{\eta}\right)^{\beta-1}e^{-\frac{x - \gamma}{\eta}^{\beta}};\ \beta > 0, \ \eta > \gamma \geq 0.
	\end{equation}
	
	Tenha ciência também que sua acumulada tem o seguinte resultado:
	\begin{equation*}
	 F(x) = 1 - e^{-\frac{x - \gamma}{\eta}^{\beta}}, 
	\end{equation*}
	 sendo $\beta,\ \eta, \ \gamma$ os parâmetros de forma, escala e locação, respectivamente.
	\end{frame}
	
	\begin{frame}
	 A distribuição Weibull tri-paramétrica é raramente usada devido à dificuldade de se estimar estes parâmetros. Portanto, o uso de SA será feito para tentar estimar estes parâmetros. Aqui o algoritmo será fundamental para maximizar a função de verossimilhança. 
	\end{frame}
	
	\begin{frame}{Estimação por EMV}
	 Temos ciência que a função de máxima verossimilhança é descrita da seguinte forma:
	 
	 \begin{equation*}
	  L(x) = \prod_{i=1}^{n} f_{x_i}(x_i \theta)
	 \end{equation*}
	 
	 Com isso, a log-verossimilhança é a seguinte:
	 
	 \begin{equation}
	  \text{Ln}(L(x_1, \dots, x_n, \beta, \eta, \gamma)) = n\text{Ln}\left(\frac{\beta}{\eta}\right) + \sum_{i=1}^{n} \left( -\left(\frac{x_i - \gamma}{\eta}\right)^{\beta} + (\beta - 1)\text{Ln}\left(\frac{x_i - \gamma}{\eta}\right) \right).
	 \end{equation}
	 
	 Para encontrar as estimativas dos parâmetros devemos realizar as derivadas parciais e igualar a zero, da seguinte forma:
	 
	 \begin{equation*}
	  \frac{\partial}{\partial \theta} Ln(L(x)) = 0
	 \end{equation*}
	 
	 Fica evidente que esta é uma função bem difícil de derivar, necessitando derivar gradiente ou mesmo aplicar métodos numéricos, portanto prosseguiremos para a ideia geral do algoritmo SA
	\end{frame}
	
	\subsection{Algoritmo Geral de Recozimento Simulado}
	\begin{frame}{Algoritmo}
	 A principal vantagem deste método perante os demais é  fato de que este, ao se empregar uma busca aleatória, \textbf{não se prende a pontos de mínimos locais}. O algoritmo aceita mudanças que aceitam a diminuição e também o aumento da função objetivo \textit{f}. O seu aumento é aceito com probabilidade: $p = e^{-\frac{\Delta}{T}}$ sendo $\theta$ o aumento e $T$ o parâmetro de controle, analogamente conhecido como temperatura do sistema. A sua implementação é simples:
	 
	 \begin{itemize}
	  \item Uma representação de soluções possíveis,
	  \item Um gerador de mudanças aleatórias nas soluções,
	  \item Um meio de avaliar as funções do problema, e
	  
	  \item Um esquema de resfriamento (annealing schedule) – uma temperatura inicial e regras para diminuí-la à medida que a busca progride.
	 \end{itemize}
	
	\end{frame}
 \noindent
 {\color{red}
\begin{verbatim}
 simulated_annealing <- function(f, S0, T0, L, alpha, T_min) {
    T <- T0; S <- S0
    S_star <- S0; f_S <- f(S)
    f_S_star <- f_S
    while (T > T_min) {
        for (n in 1:L) {
            S_n <- gerar_vizinho(S) 
            f_S_n <- f(S_n); Delta <- f_S_n - f_S
            if (Delta <= 0) {
                S <- S_n; f_S <- f_S_n
            } else {
                p <- exp(-Delta / T)
                if (runif(1) < p) {
                    S <- S_n; f_S <- f_S_n
                }}}
        if (f_S < f_S_star) {
            S_star <- S; f_S_star <- f_S
        }
        T <- T * alpha
    } 
    return(list(best_solution = S_star, best_cost = f_S_star))}

\end{verbatim}
	}
	\begin{frame}
	 Por ser um algoritmo meta-heurístico, urge a importância de o mesmo ter a possibilidade de aceitar uma solução pior, evitando que o algoritmo caia em um máximo local e permaneça por lá. Claramente a probabilidade de aceitar uma pior solução diminui na mesma medida que a temperatura cai em cada um dos outros ciclos. A performance do algoritmo portanto é definida dentro de parâmetros de controle, sendo eles:
	\begin{itemize}
	\item A temperatura inicial ($T_0$) deve ser grande o suficiente para que na primeira iteração do algoritmo,a probabilidade de aceitação de um cenário ruim seja, pelo menos 80\%;
	\item É utilizada a função de redução, sendo esta geométrica comprrendida por: $T_i = CT_{i-1}$, onde $C<1$, constante e se encontra usualmente entre 0.75 e 0.95;
	\item O tamanho de cada nível de temperatura determina o número de soluções geradas em uma certa temperatura T;
	\item O critério de parada define quando o sistema encontrou o nível de eneergia desejado. Em suma, isso define:
	\begin{itemize}
	 \item O totald e números de soluções geradas.
	 \item A temperatuira a cada nível de energia desejado.
	 \item A razão de aceitação (entre o número de soluções aceitadas e geradas).
	\end{itemize}
	 	\end{itemize}
	\end{frame}
	
	
	\begin{frame}[fragile]
    \frametitle{Algoritmo SA para Maximização da Verossimilhança}
    \framesubtitle{Estimativa de $\beta, \eta, \gamma$}
    

    \centering

    \scriptsize 
    

    \begin{tabular}{|c|p{7.5cm}|}
        \hline
        \textbf{Passo} & \textbf{Ação / Descrição} \\
        \hline
        1 & \textbf{Obter Amostra} aleatória (tamanho grande o suficiente). \\
        \hline
        2 & \textbf{Determinar Parâmetros de Controle} ($T_0, T_f, C, I$). \\
        \hline
        3 & \textbf{Gerar Solução Inicial} aleatória ($a, b, c$). \\
        \hline
        4 & \textbf{Computar Verossimilhança Inicial} ($L$) na solução $(a, b, c)$. \\
        \hline
        \multicolumn{2}{|c|}{\textbf{Loop Externo (Resfriamento)}} \\
        \hline
        5 & \textbf{Enquanto} $T > T_0$: 
            \[ T \leftarrow C T \] \\
        \hline
        \multicolumn{2}{|c|}{\textbf{Loop Interno (Equilíbrio)}} \\
        \hline
        6 & \textbf{Para} $i = 1 \text{ a } I$: \\
        \hline
        6.1 & \textbf{Gerar Vizinho:} Gerar $a_1, b_1, c_1$ vizinhos de $a, b, c$. \\
        \hline
        6.2 & \textbf{Calcular Verossimilhança Vizinha} ($L_0$) em $(a_1, b_1, c_1)$. \\
        \hline
        6.3 & \textbf{Avaliar Parâmetros (Critério de Metropolis):} \\
        \hline
        6.3.1 & \textbf{Se} $L_0 > L$ (Melhora):
            \[ a \leftarrow a_1, b \leftarrow b_1, c \leftarrow c_1, \text{ e } L \leftarrow L_0 \] \\
        \hline
        6.3.2 & \textbf{Senão} (Piora): Gerar $u \sim \text{Uni}(0, 1)$. \\
        \hline
    \end{tabular}
    
\end{frame}
 \begin{frame}
 \begin{tabular}{|c|p{7.5cm}|}
  \hline
  6.3.2.1 & \textbf{Se} $u < e^{\frac{(F_0 - F)}{T}}$ (Aceitação Probabilística):
            \[ a \leftarrow a_1, b \leftarrow b_1, c \leftarrow c_1 \] \\
        \hline
        \multicolumn{2}{|c|}{(Fim do Loop Interno)} \\
        \hline
        \multicolumn{2}{|c|}{(Fim do Loop Externo)} \\
        \hline
        7 & \textbf{Resultado:} Imprimir $a, b, c \text{ e } L$. \\
        \hline
        \multicolumn{2}{|c|}{\footnotesize \textbf{Nota:} $a, b, c$ são estimativas de $\beta, \eta, \gamma$. $F$ é uma função de custo/energia (ex: $-\ln(L)$).} \\
        \hline
        \end{tabular}
        
 \end{frame}
	
	\subsection{Implementação}
		
\begin{frame}[fragile]{Código implementado}
	\noindent
	{\color{blue} \footnotesize
		\begin{verbatim}
loglik_weibull3 <- function(params, x) {
  b <- params[1] # beta (forma)
  g <- params[2] # eta (escala)
  c <- params[3] # gamma (locação)
  
  if (b <= 0 || g <= 0 || any(x <= c)) {
    return(-Inf)
  }
  
  z <- (x - c) / g
  # Fórmula Log-Verossimilhança (Eq 7 do artigo)
  ll <- length(x) * (log(b) - log(g)) + sum((b - 1) * log(z) - (z^b))
  return(ll)
}

    \end{verbatim}
	}
	\end{frame}
	
	
	\noindent
	{\color{blue} \footnotesize
	 	\begin{verbatim}
	propose_neighbor <- function(curr, x, sds = c(0.1, 0.1, 0.1)) {
  attempt <- 0
  min_x <- min(x)
  repeat {
    attempt <- attempt + 1
    b1 <- curr[1] + rnorm(1, mean = 0, sd = sds[1])
    g1 <- curr[2] + rnorm(1, mean = 0, sd = sds[2])
    c1 <- curr[3] + rnorm(1, mean = 0, sd = sds[3])
    if (b1 <= 0) {
      b1 <- abs(b1) + 1e-6
    }
    if (g1 <= 0) {
      g1 <- abs(g1) + 1e-6
    }
    if (c1 < min_x - 1e-8) {
      return(c(b1, g1, c1))
    }
    if (attempt > 50) {
      sds[3] <- sds[3] * 1.5
    }
    if (attempt > 1000) {
      return(c(b1, g1, min_x - 1e-6))}}}
      
      sa_weibull3 <- function(
  x,
  init = NULL,
  T0 = 100,
  Tf = 0.001,
  cooling_rate = 0.99,
  L = 5,
  sds = c(0.1, 0.1, 0.1)
) {
  n <- length(x)
  # Lógica de Inicialização (Simplificada)
  if (is.null(init)) {
    c0 <- min(x) - 0.1 * sd(x)
    if (c0 >= min(x)) {
      c0 <- min(x) - 1e-3
    }
    y <- x - c0
    y[y <= 0] <- min(y[y > 0]) * 0.1
    y_sorted <- sort(y)
    prob_rank <- (1:n) / (n + 1)
    ln_y <- log(y_sorted)
    ln_ln <- log(-log(1 - prob_rank))
    fit <- try(lm(ln_ln ~ ln_y), silent = TRUE)
    if (inherits(fit, "try-error")) {
      init <- c(1.5, sd(x), min(x) - 0.1)
    } else {
      b0 <- fit$coefficients[2]
      g0 <- exp(-fit$coefficients[1] / b0)
      init <- c(b0, g0, c0)
      if (
        !is.finite(b0) || !is.finite(g0) || !is.finite(c0) || b0 <= 0 || g0 <= 0
      ) {
        init <- c(1.5, sd(x), min(x) - 0.1)
      }
    }
  }
  curr <- init
  curr_ll <- loglik_weibull3(curr, x)
  best <- curr
  best_ll <- curr_ll
  T <- T0
  history <- data.frame(eval = 1, ll = curr_ll)
  eval_count <- 1

  # Loop SA
  while (T > Tf) {
    for (i in 1:L) {
      prop <- propose_neighbor(curr, x, sds = sds)
      prop_ll <- loglik_weibull3(prop, x)
      if (prop_ll > curr_ll) {
        curr <- prop
        curr_ll <- prop_ll
      } else {
        delta_ll <- prop_ll - curr_ll

        if (runif(1) < exp(delta_ll / T)) {
          curr <- prop
          curr_ll <- prop_ll
        }
      }
      if (curr_ll > best_ll) {
        best <- curr
        best_ll <- curr_ll
      }
      eval_count <- eval_count + 1
      history <- rbind(history, data.frame(eval = eval_count, ll = curr_ll))
    }
    T <- cooling_rate * T}
  return(list(
    best_params = best,
    best_ll = best_ll,
    history = history,
    evals = eval_count
  ))
}
 
    \end{verbatim}
	}

\begin{frame}[fragile] % Use [fragile] se o seu slide tiver muito código R ou comandos especiais
\frametitle{Resultados da Estimação por Recozimento Simulado}
\framesubtitle{Weibull ($\beta =2, \ \eta =2,\ \gamma = 2)$}

\begin{table}[ht]
\centering
% Opcional: Reduz o tamanho da fonte para caber se for usado em slide
\tiny
% Opcional: Força a tabela a ter a largura da linha
\resizebox{\linewidth}{!}{%
    \begin{tabular}{lrrrr} % Formato {lrrrr} para 1 coluna de rótulo e 4 de dados
        \toprule
        % Parâmetros (2, 2, 2)
        \multicolumn{1}{c}{\textbf{Weibull parameters}} & \multicolumn{4}{c}{\textbf{(2, 2, 2)}}\\\\
        \hline
        % Cabeçalhos das colunas
        \textbf{Measure} & \textbf{2500} & \textbf{1000} & \textbf{500} & \textbf{100} \\
        \midrule
        % Linha de parâmetros estimados, corrigindo os comandos de símbolo
        Estimated parameters ($\beta$, $\eta$, $\gamma$) & (1.9829, 2.0162, 2.0177) & (1.9374, 2.0165, 1.9982) & (2.0222, 1.9569, 2.0512) & (1.8491, 2.0029, 2.1073) \\
        % Likelihood nos valores reais
        Likelihood function at the real values & -3254.7346 & -1326.3027 & -631.2297 & -136.3167 \\
        % Likelihood nos valores estimados
        Likelihood function at the estimated value & -3252.9574 & -1324.7026 & -629.9147 & -135.1983 \\
        % Tempo de execução
        Run time (s) & 7.2342 & 6.0214 & 5.9584 & 6.8564 \\
        \bottomrule
    \end{tabular}%
}
\caption{Tabela de Resultados para o Exemplo 1: Weibull (2, 2, 2) via SA}
\label{tab:sa_weibull_results_ex1}
\end{table}

\end{frame}

\begin{frame}[fragile] % Use [fragile] se o seu slide tiver muito código R ou comandos especiais
\frametitle{Resultados da Estimação por Recozimento Simulado}
\framesubtitle{Weibull ($\beta =2, \ \eta =2,\ \gamma = 2)$}
\begin{figure}[h]
			\centering
			\includegraphics[width=0.8\textwidth]{sa_weibull_convergencia_exemplo1.png}
		\end{figure}
\end{frame}

\begin{frame}[fragile] % Use [fragile] se o seu slide tiver muito código R ou comandos especiais
\frametitle{Resultados da Estimação por Recozimento Simulado}
\framesubtitle{Weibull ($\beta =3, \ \eta =5,\ \gamma = 7)$}

\begin{table}[ht]
\centering
% Opcional: Reduz o tamanho da fonte para caber se for usado em slide
\tiny
% Opcional: Força a tabela a ter a largura da linha
\resizebox{\linewidth}{!}{%
    \begin{tabular}{lrrrr} % Formato {lrrrr} para 1 coluna de rótulo e 4 de dados
        \toprule
        % Parâmetros (3, 5, 7)
        \multicolumn{1}{c}{\textbf{Weibull parameters}} & \multicolumn{4}{c}{\textbf{(3, 5, 7)}}\\\\
        \hline
        % Cabeçalhos das colunas
        \textbf{Measure} & \textbf{2500} & \textbf{1000} & \textbf{500} & \textbf{100} \\
        \midrule
        % Linha de parâmetros estimados
        Estimated parameters ($\beta$, $\eta$, $\gamma$) & (2.9609, 4.9583, 7.0445) & (2.9254, 4.8563, 7.0478) & (3.2833, 5.1335, 6.8745) & (2.7800, 4.5803, 7.3323) \\
        % Likelihood nos valores reais
        Likelihood function at the real values & -4741.6982 & -1889.3753 & -927.9355 & -186.2723 \\
        % Likelihood nos valores estimados
        Likelihood function at the estimated value & -4741.5966 & -1887.9379 & -926.3573 & -185.9554 \\
        % Tempo de execução
        Run time (s) & 6.6395 & 6.0145 & 5.7210 & 5.6069 \\
        \bottomrule
    \end{tabular}%
}
\caption{Tabela de Resultados para o Exemplo 2: Weibull (3, 5, 7) via SA}
\label{tab:sa_weibull_results_ex2}
\end{table}

\end{frame}

\begin{frame}[fragile] % Use [fragile] se o seu slide tiver muito código R ou comandos especiais
\frametitle{Resultados da Estimação por Recozimento Simulado}
\framesubtitle{Weibull ($\beta =3, \ \eta =5,\ \gamma = 7)$}
\begin{figure}[h]
			\centering
			\includegraphics[width=0.8\textwidth]{sa_weibull_convergencia_exemplo2.png}
		\end{figure}
\end{frame}


\begin{frame}[fragile] % Use [fragile] se o seu slide tiver muito código R ou comandos especiais
\frametitle{Resultados da Estimação por Recozimento Simulado}
\framesubtitle{Weibull ($\beta =8, \ \eta =4,\ \gamma = 6)$}

\begin{table}[ht]
\centering
% 1. Aplica o menor tamanho de fonte (opcional, dependendo do conteúdo)
\tiny
% 2. Usa resizebox para forçar a largura total da tabela (opcional, dependendo do conteúdo)
\resizebox{\linewidth}{!}{%
    \begin{tabular}{lrrrr} % Formato {lrrrr} para 1 coluna de rótulo e 4 de dados
        \toprule
        \multicolumn{1}{c}{\textbf{Weibull parameters}} & \multicolumn{4}{c}{\textbf{(8, 4, 6)}}\\\\ % Duas quebras de linha para simular o espaçamento do primeiro slide
        \hline
        \textbf{Measure} & \textbf{2500} & \textbf{1000} & \textbf{500} & \textbf{100} \\
        \midrule
        Estimated parameters ($\beta$, $\eta$, $\gamma$) & (7.3183, 3.7274, 6.2453) & (7.4927, 3.8168, 6.2023) & (5.9484, 3.0631, 6.9459) & (4.2302, 1.9655, 8.0114) \\
        Likelihood function at the real values & -2062.5242 & -831.9918 & -411.7484 & -70.3477 \\
        Likelihood function at the estimated value & -2059.4080 & -830.5465 & -410.4577 & -67.7416 \\
        Run time (s) & 6.6368 & 6.0142 & 6.4628 & 5.9164 \\
        \bottomrule
    \end{tabular}%
}
\caption{Tabela de Resultados para o Exemplo 3: Weibull (8, 4, 6) via SA}
\label{tab:sa_weibull_results_ex3}
\end{table}
\end{frame}

\begin{frame}[fragile] % Use [fragile] se o seu slide tiver muito código R ou comandos especiais
\frametitle{Resultados da Estimação por Recozimento Simulado}
\framesubtitle{Weibull ($\beta =8, \ \eta =4,\ \gamma = 6)$}
\begin{figure}[h]
			\centering
			\includegraphics[width=0.8\textwidth]{sa_weibull_convergencia_exemplo3.png}
		\end{figure}
\end{frame}
	
	
	\end{document}