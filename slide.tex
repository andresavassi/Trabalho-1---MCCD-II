\documentclass{beamer} % Define a classe do documento como Beamer

\usepackage{listings}
\usepackage{xcolor} % necessário para as cores no listings
\usepackage{listings}
\usepackage{xcolor}
\usepackage{graphicx}
\usepackage[percent]{overpic}


\lstset{
	language=R,
	basicstyle=\ttfamily\footnotesize,
	keywordstyle=\color{blue},
	commentstyle=\color{gray},
	stringstyle=\color{red},
	frame=single,
	breaklines=true,
	columns=flexible,
	showstringspaces=false,
	tabsize=2
}

\usetheme{Ilmenau}   
\graphicspath{{figuras/}}  

% Informações do Título
\title{Estudo sobre modelos log-lineares Poisson: \\ Verificação de estimativa pontual e intervalar (tipo link) para as médias}
\author{André Savassi \\ Kenzo Bontempo \\ Márcio Antônio}
\institute{UFMG}
\date{01/10/2025} % Data da apresentação

% CONFIGURAÇÃO PERSONALIZADA DO RODAPÉ
\setbeamertemplate{footline}{
	\leavevmode%
	\hbox{%
		% ESQUERDA: "grupo 2"
		\begin{beamercolorbox}[wd=.33\paperwidth,ht=2.25ex,dp=1ex,left]{author in head/foot}%
			\hspace{0.5cm}\textbf{Grupo 2}%
		\end{beamercolorbox}%
		% CENTRO: "MCCD -2"
		\begin{beamercolorbox}[wd=.34\paperwidth,ht=2.25ex,dp=1ex,center]{title in head/foot}%
			\textbf{MCCD - II}%
		\end{beamercolorbox}%
		% DIREITA: Número da página (opcional)
		\begin{beamercolorbox}[wd=.33\paperwidth,ht=2.25ex,dp=1ex,right]{date in head/foot}%
			\textbf{01/10/2025}
			\insertpagenumber\hspace{0.5cm}%
	\end{beamercolorbox}}%
	\vskip0pt%
}

% Início do Documento
\begin{document}
	
	% Slide de Título
	\begin{frame}
		\titlepage % Gera o slide de título com as informações acima
	\end{frame}
	
	% Sumário (Opcional, mas recomendado)
	\begin{frame}{Sumário}
		\tableofcontents % Gera um sumário automático baseado nas seções
	\end{frame}
	
	% Primeira Seção
\section{Introdução}

\begin{frame}{Introdução}
	\begin{itemize}
		\item \textbf{Modelos log-lineares Poisson:} principais ferramentas para modelar variáveis de contagem.
		\begin{itemize}
			\item Aplicações em epidemiologia, ciências sociais e controle de qualidade.
		\end{itemize}
		
		\item A média é expressa como \textbf{função exponencial das covariáveis}, garantindo positividade das estimativas.
		
		\item \textbf{Objetivo do estudo:}
		\begin{itemize}
			\item Verificar \textbf{estimativas pontuais e intervalares} da média do modelo Poisson.
			\item Utilizar \textbf{simulações de Monte Carlo} para avaliar desempenho dos estimadores.
		\end{itemize}
	\end{itemize}
\end{frame} % ✅ fecha o primeiro frame certinho


\begin{frame}{Introdução}
	\begin{itemize}
		\item \textbf{Métricas analisadas:}
		\begin{itemize}
			\item Viés relativo
			\item Erro padrão
			\item Taxa de cobertura dos intervalos de confiança
		\end{itemize}
		
		\item \textbf{Comparações adicionais:}
		\begin{itemize}
			\item Distribuições \textbf{Binomial Negativa} e \textbf{Bell}, avaliando robustez e sobredispersão.
		\end{itemize}
	\end{itemize}
\end{frame}

	% Slide de Conteúdo


	\section{Distribuições}
	\begin{frame}{Gerando amostras de distribuições - Poisson}
		\subtitle{Distribuição Poisson}
		Primeiramente, iremos amostrar da Distribuição Poisson, que possui a função densidade descrita por:
		$$f(y; \lambda) = \frac{e^{-\lambda} \lambda^y}{y!}, \ y = 0,1,\dots,n$$
		
		Referencia-se que: $\mathbb{E}(Y) = Var(Y) = \lambda$
 		
	\end{frame}
	
	% Segunda Seção e outro Slide

	
\begin{frame}[fragile]{Amostrador Poisson via Aceitação e Rejeição}
	\begin{columns}
		
		\begin{column}{0.85\textwidth}
			\begin{lstlisting}[language=R]
	poisson_ar_1 <- function(lambda) {
		f_pois <- function(k, lambda) {
			return(exp(k * log(lambda) - lambda - lgamma(k + 1)))
		}
		fmax <- f_pois(floor(lambda), lambda)
		kmax <- qpois(0.9999, lambda = 5)
		repeat {
			y <- sample(0:kmax, 1)
			fy <- f_pois(y, lambda)
			gy <- 1 / (kmax + 1)
			u <- runif(1)
			if (u < fy / (fmax)) {
				return(y)
			}}}
	poisson_ar <- function(lambda, n = 1) {
		replicate(n, poisson_ar_1(lambda))
	}		
			\end{lstlisting}
		\end{column}
		
		\begin{column}{0.15\textwidth}
			\href{https://github.com/andresavassi/Trabalho-1---MCCD-II/blob/main/rpoisson.R}{\beamergotobutton{Código.R}}
			
			\vspace{0.5cm}

		\end{column}
		
	\end{columns}
\end{frame}

	\begin{frame}{Gráficos - Poisson AR}
	\begin{figure}[h]
		\centering
		\includegraphics[width=0.8\textwidth]{graf_poisson.png}
	\end{figure}
\end{frame}





	\begin{frame}{Gerando amostras de distribuições - Binomial Negativa}
		Segue que a densidade da distribuição Binomial Negativa é descrita por:
		$$f(y; r,p) = \binom{y-1}{r-1} p^r(1-p)^{y-r}$$
		\ onde $r$ é o total de sucessos e $p$ é a probabilidade desses sucessos.
		Note que $Y \sim BN(r,p)$ possui esperança e variância a seguir:
		$$
		\mathbb{E}(Y) = \frac{r}{p}\ ;\ Var(Y) = \frac{r(1-p)}{p^2}
		$$
		
		
	\end{frame}
	
	
	\begin{frame}[fragile]{Amostrador BN via Amostragem por Importância}
			\begin{columns}
			
			\begin{column}{0.85\textwidth}
				\begin{lstlisting}[language=R]
rnegbinom <- function(m, n, mu, theta) {
	f_negbinom <- function(x, mu, theta) {
		log_f <- lgamma(x + theta) -
		lgamma(theta) -
		lgamma(x + 1) +
		theta * log(theta / (mu + theta)) +
		x * log(mu / (mu + theta))
		return(exp(log_f))
	}
  g_pois <- function(k, lambda) {
	return(exp(k * log(lambda) - lambda - lgamma(k + 1)))
}

				\end{lstlisting}
			\end{column}
			
		
			\begin{column}{0.15\textwidth}
				\href{https://github.com/andresavassi/Trabalho-1---MCCD-II/blob/main/rnegbin.R}{\beamergotobutton{Código.R}}
				
				\vspace{0.5cm}
				
			\end{column}
			
		\end{columns}
		\end{frame}
		
		\begin{frame}[fragile]{Amostrador BN via Amostragem por Importância}
			\begin{lstlisting}
  source("rpoisson.R")

y <- rpois(n = m, lambda = 0.9 * mu)

pesos <- f_negbinom(y, mu = mu, theta = theta) / g_pois(y, lambda = 0.9 * mu)
pesos <- pesos / sum(pesos)

amostra <- sample(y, n, F, pesos)

return(amostra)
}
			\end{lstlisting}
			
			\end{frame}
		
			\begin{frame}{Gráficos - BN Amostragem por Importância}
			\begin{figure}[h]
				\centering
				\includegraphics[width=0.8\textwidth]{graf_bn.png}
			\end{figure}
		\end{frame}
		
		
	\begin{frame}{Gerando amostras de distribuições - Bell}
		Segue que a densidade da distribuição Bell é descrita por:
		$$
		f(y; \theta) = \frac{\theta e^{e^{\theta} + 1}}{y!} B_Y,\ y = 0, 1, \dots, n; \ \theta > 0
		$$
		
		Onde o termo $B_Y$ corresponde ao número de Bell, dados por:
		
		$$
		B_n= \frac{1}{e} \sum_{k = 0}^{\infty} \frac{k^n}{k!}, \text{iniciando com}\ B_0 = B_1 = 1
		$$
		
		A média e variância de $Y \sim Bell(\theta)$ é:
		$$
		\mathbb{E}(Y) = \theta e^{\theta}\ ;\ 
		Var(Y) = \theta(1+\theta)e^{\theta}
		$$
		
	\end{frame}
	
	
		\begin{frame}[fragile]{Amostrador Bell via Transformação Inversa}
		\begin{columns}
			
			\begin{column}{0.85\textwidth}
				\begin{lstlisting}[language=R]
		# Importado de bellreg
		dbell <- function(x, theta, log = FALSE) {
			Bx <- c()
			for (i in 1:length(x)) {
				Bx[i] <- numbers::bell(x[i])
			}
			lf <- x * log(theta) - exp(theta) + 1 + log(Bx) - lgamma(x + 1)
			if (log == TRUE) {
				return(lf)
			} else {
				return(exp(lf))
			}
		}		
				\end{lstlisting}
			\end{column}
			
			\begin{column}{0.15\textwidth}
				\href{https://github.com/andresavassi/Trabalho-1---MCCD-II/blob/main/rbell.R}{\beamergotobutton{Código.R}}
				
				\vspace{0.5cm}
				
			\end{column}
			
		\end{columns}
	\end{frame}
	
	\begin{frame}[fragile]{Amostrador Bell via Transformação Inversa}
		\begin{lstlisting}		
		rbell_aux <- function(theta) {
			sapply(theta, function(t) {
				u <- runif(1, 0, 1)
				i <- 0
				pr <- dbell(0, t); Fx <- pr
				while (u >= Fx) {
					i <- i + 1
					pr <- dbell(i, t)
					Fx <- Fx + pr}
				return(i)
			})}
		rbell <- function(n, theta) {
			if (length(theta) == 1) {
				return(replicate(n, expr = rbell_aux(theta), simplify = TRUE))
			} else {
				return(rbell_aux(theta))
			}}
		\end{lstlisting}
		
		\end{frame}
	
	\begin{frame}{Gráficos - Bell Inversa}
	\begin{figure}[h]
		\centering
		\includegraphics[width=0.8\textwidth]{graf_bell.png}
	\end{figure}
\end{frame}

	

	\section{Implementação}
	
	\begin{frame}[fragile]{Implementação do Algoritmo}
		\begin{itemize}
			\item 1. Gere covariáveis contínuas e discretas e as organize na matriz $X_{(n+m)x p}$;
			\item 2. Escolha arbitrariamente um vetor $\beta$;
			\item 3. Calcule $\mu_i = exp(\eta_i)$, onde $\eta_i = X_i^T \beta$, $\forall i = 1, \dots, n+m$;
			\item 4. Gere $Y_i$ de Poisson($\mu_i$), $\forall i = 1, \dots,n$;
			\item 5. Avaliar o modelo usando as $n$ primeiras amostras;
			\item 6 .Calcular estimação pontual e intervalar da média $mu_i$ para as $m$ últimas amostras;
			\item 7. Compare os resultados com os $\mu_i$ gerados no item(3), a partir do víes relativo e CP cobertura;
			\item 8. Repita de 1 até 7 até alcançar o número de repetições desejadas.
		\end{itemize}
		\end{frame}
	\begin{frame}[fragile]{Monte Carlo - Poisson}
		\href{https://github.com/andresavassi/Trabalho-1---MCCD-II/blob/main/mc_poisson.R}{\beamergotobutton{Código.R}}

	\end{frame}
	
		
		\begin{frame}[fragile]{Tabelas - MC Poisson}
			\resizebox{\textwidth}{!}{%
		\begin{tabular}{c|c|c|c|c|c|c|c|c|c}
			\hline
			Tamanho da Amostra & Parâmetro & Valor Real & Estimativa Média & Erro Padrão & Desvio Padrão & Viés Relativo & Limite Inferior & Limite Superior & Cobertura\\
			\hline
			100 & y1 & 3.362 & 3.354 & 0.014 & 0.386 & -0.225 & 2.673 & 4.211 & 0.950\\
			\hline
			100 & y10 & 3.798 & 3.790 & 0.010 & 0.378 & -0.210 & 3.119 & 4.607 & 0.958\\
			\hline
			100 & y2 & 3.491 & 3.492 & 0.020 & 0.481 & 0.029 & 2.664 & 4.581 & 0.952\\
			\hline
			100 & y3 & 5.447 & 5.497 & 0.025 & 0.914 & 0.934 & 4.036 & 7.494 & 0.937\\
			\hline
			100 & y4 & 2.513 & 2.507 & 0.014 & 0.285 & -0.259 & 1.998 & 3.146 & 0.952\\
			\hline
			100 & y5 & 3.980 & 3.997 & 0.029 & 0.682 & 0.424 & 2.873 & 5.565 & 0.949\\
			\hline
			100 & y6 & 4.220 & 4.217 & 0.011 & 0.440 & -0.057 & 3.447 & 5.162 & 0.950\\
			\hline
			100 & y7 & 3.097 & 3.093 & 0.012 & 0.334 & -0.151 & 2.501 & 3.825 & 0.956\\
			\hline
			100 & y8 & 4.512 & 4.514 & 0.021 & 0.658 & 0.039 & 3.401 & 5.998 & 0.959\\
			\hline
			100 & y9 & 3.845 & 3.840 & 0.011 & 0.392 & -0.148 & 3.143 & 4.692 & 0.958\\
			\hline
		\end{tabular}
	}
		\end{frame}
	
	
	\begin{frame}{Gráficos - MC Poisson}
		\begin{figure}[h]
			\centering
			\includegraphics[width=0.8\textwidth]{grafico_poisson1.png}
		\end{figure}
	\end{frame}
	
		\begin{frame}{Gráficos - MC Poisson}
		\begin{figure}[h]
			\centering
			\includegraphics[width=0.8\textwidth]{grafico_poisson2.png}
		\end{figure}
	\end{frame}
	
	
	\begin{frame}[fragile]{Monte Carlo - Binomial Negativa}
		\href{https://github.com/andresavassi/Trabalho-1---MCCD-II/blob/main/mc_negbin.R}{\beamergotobutton{Código.R}}

	\end{frame}
	
	\begin{frame}[fragile]{Tabelas - MC Binomial Negativa}
		\resizebox{\textwidth}{!}{%
			\begin{tabular}{c|c|c|c|c|c|c|c|c|c}
				\hline
				Tamanho da Amostra & Parâmetro & Valor Real & Estimativa Média & Erro Padrão & Desvio Padrão & Viés Relativo & Limite Inferior & Limite Superior & Cobertura\\
				\hline
				100 & y1 & 3.362 & 3.142 & 0.015 & 0.562 & -6.546 & 2.485 & 3.975 & 0.783\\
				\hline
				100 & y10 & 3.798 & 3.494 & 0.011 & 0.560 & -8.012 & 2.853 & 4.280 & 0.734\\
				\hline
				100 & y2 & 3.491 & 3.227 & 0.022 & 0.711 & -7.559 & 2.438 & 4.277 & 0.775\\
				\hline
				100 & y3 & 5.447 & 4.978 & 0.028 & 1.286 & -8.607 & 3.612 & 6.867 & 0.782\\
				\hline
				100 & y4 & 2.513 & 2.397 & 0.015 & 0.399 & -4.626 & 1.897 & 3.029 & 0.820\\
				\hline
				100 & y5 & 3.980 & 3.715 & 0.032 & 0.994 & -6.660 & 2.642 & 5.231 & 0.783\\
				\hline
				100 & y6 & 4.220 & 3.853 & 0.012 & 0.654 & -8.686 & 3.123 & 4.757 & 0.718\\
				\hline
				100 & y7 & 3.097 & 2.902 & 0.013 & 0.457 & -6.317 & 2.331 & 3.614 & 0.796\\
				\hline
				100 & y8 & 4.512 & 4.120 & 0.024 & 0.966 & -8.696 & 3.071 & 5.535 & 0.760\\
				\hline
				100 & y9 & 3.845 & 3.532 & 0.012 & 0.581 & -8.142 & 2.869 & 4.351 & 0.731\\
				\hline
			\end{tabular}
		}
	\end{frame}
	
	
	
		\begin{frame}{Gráficos - MC Binomial Negativa}
		\begin{figure}[h]
			\centering
			\includegraphics[width=0.8\textwidth]{grafico_negbin1.png}
		\end{figure}
	\end{frame}
	
			\begin{frame}{Gráficos - MC Binomial Negativa}
		\begin{figure}[h]
			\centering
			\includegraphics[width=0.8\textwidth]{grafico_negbin2.png}
		\end{figure}
	\end{frame}


\begin{frame}[fragile]{Monte Carlo - Bell}
		\href{https://github.com/andresavassi/Trabalho-1---MCCD-II/blob/main/mc_bell.R}{\beamergotobutton{Código.R}}
				\end{frame}
			

\begin{frame}[fragile]{Tabelas - MC Bell}
	\resizebox{\textwidth}{!}{%
		\begin{tabular}{c|c|c|c|c|c|c|c|c|c}
			\hline
			Tamanho da Amostra & Parâmetro & Valor Real & Estimativa Média & Erro Padrão & Desvio Padrão & Viés Relativo & Limite Inferior & Limite Superior & Cobertura\\
			\hline
			100 & y1 & 3.362 & 3.374 & 0.014 & 0.578 & 0.367 & 2.691 & 4.233 & 0.820\\
			\hline
			100 & y10 & 3.798 & 3.785 & 0.010 & 0.568 & -0.326 & 3.116 & 4.601 & 0.812\\
			\hline
			100 & y2 & 3.491 & 3.482 & 0.020 & 0.715 & -0.243 & 2.658 & 4.568 & 0.823\\
			\hline
			100 & y3 & 5.447 & 5.531 & 0.026 & 1.370 & 1.547 & 4.064 & 7.534 & 0.796\\
			\hline
			100 & y4 & 2.513 & 2.530 & 0.014 & 0.422 & 0.676 & 2.018 & 3.174 & 0.824\\
			\hline
			100 & y5 & 3.980 & 4.032 & 0.029 & 0.972 & 1.321 & 2.904 & 5.607 & 0.823\\
			\hline
			100 & y6 & 4.220 & 4.206 & 0.011 & 0.666 & -0.319 & 3.438 & 5.149 & 0.816\\
			\hline
			100 & y7 & 3.097 & 3.113 & 0.012 & 0.506 & 0.506 & 2.519 & 3.848 & 0.815\\
			\hline
			100 & y8 & 4.512 & 4.511 & 0.022 & 0.964 & -0.020 & 3.403 & 5.989 & 0.822\\
			\hline
			100 & y9 & 3.845 & 3.830 & 0.011 & 0.586 & -0.388 & 3.136 & 4.680 & 0.816\\
			\hline
		\end{tabular}
	}
\end{frame}


			\begin{frame}{Gráficos - MC Bell}
		\begin{figure}[h]
			\centering
			\includegraphics[width=0.8\textwidth]{grafico_bell1.png}
		\end{figure}
		\end{frame}
		
				\begin{frame}{Gráficos - MC Bell}
			\begin{figure}[h]
				\centering
				\includegraphics[width=0.8\textwidth]{grafico_bell2.png}
			\end{figure}
		\end{frame}
\end{document}
