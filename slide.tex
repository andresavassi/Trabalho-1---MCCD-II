\documentclass{beamer} % Define a classe do documento como Beamer

\usepackage{listings}
\usepackage{xcolor} % necessário para as cores no listings
\usepackage{listings}
\usepackage{xcolor}
\usepackage[percent]{overpic}


\lstset{
	language=R,
	basicstyle=\ttfamily\footnotesize,
	keywordstyle=\color{blue},
	commentstyle=\color{gray},
	stringstyle=\color{red},
	frame=single,
	breaklines=true,
	columns=flexible,
	showstringspaces=false,
	tabsize=2
}

\usetheme{Ilmenau}   
\graphicspath{{figuras/}}  

% Informações do Título
\title{Estudo sobre modelos log-lineares Poisson: \\ Verificação de estimativa pontual e intervalar (tipo link) para as médias}
\author{André Savassi \\ Kenzo Bontempo \\ Márcio Antônio}
\institute{UFMG}
\date{01/10/2025} % Data da apresentação

% CONFIGURAÇÃO PERSONALIZADA DO RODAPÉ
\setbeamertemplate{footline}{
	\leavevmode%
	\hbox{%
		% ESQUERDA: "grupo 2"
		\begin{beamercolorbox}[wd=.33\paperwidth,ht=2.25ex,dp=1ex,left]{author in head/foot}%
			\hspace{0.5cm}\textbf{Grupo 2}%
		\end{beamercolorbox}%
		% CENTRO: "MCCD -2"
		\begin{beamercolorbox}[wd=.34\paperwidth,ht=2.25ex,dp=1ex,center]{title in head/foot}%
			\textbf{MCCD - II}%
		\end{beamercolorbox}%
		% DIREITA: Número da página (opcional)
		\begin{beamercolorbox}[wd=.33\paperwidth,ht=2.25ex,dp=1ex,right]{date in head/foot}%
			\textbf{01/10/2025}
			\insertpagenumber\hspace{0.5cm}%
	\end{beamercolorbox}}%
	\vskip0pt%
}

% Início do Documento
\begin{document}
	
	% Slide de Título
	\begin{frame}
		\titlepage % Gera o slide de título com as informações acima
	\end{frame}
	
	% Sumário (Opcional, mas recomendado)
	\begin{frame}{Sumário}
		\tableofcontents % Gera um sumário automático baseado nas seções
	\end{frame}
	
	% Primeira Seção
	\section{Introdução}
	
	% Slide de Conteúdo
	\begin{frame}{Introdução} % Título do slide vai aqui
		%\framesubtitle{Subtítulo (Opcional)} % Subtítulo opcional
		
		Aqui vai o conteúdo do seu slide.
		
		\begin{itemize}
			\item Este é um item de lista.
			\item<2-> Este item aparece no segundo "clique" (sobreposição).
			\item Outro item importante.
		\end{itemize}
		
		\begin{block}{Título de um Bloco}
			Texto dentro de um bloco destacado.
		\end{block}
	\end{frame}
	
	\section{Distribuições}
	\begin{frame}{Gerando amostras de distribuições - Poisson}
		\subtitle{Distribuição Poisson}
		Primeiramente, iremos amostrar da Distribuição Poisson, que possui a função densidade descrita por:
		$$f(y; \lambda) = \frac{e^{-\lambda} \lambda^y}{y!}, \ y = 0,1,\dots,n$$
		
		Referencia-se que: $\mathbb{E}(Y) = Var(Y) = \lambda$
 		
	\end{frame}
	
	% Segunda Seção e outro Slide

	
\begin{frame}[fragile]{Amostrador Poisson via Transformação Inversa}
	\begin{columns}
		
		\begin{column}{0.85\textwidth}
			\begin{lstlisting}[language=R]
	library(dplyr)
	library(kableExtra)
	rpois_aux <- function(lambda) {
		u <- runif(1, 0, 1)
		i <- 0
		pr <- exp(-lambda)
		Fx = pr
		while (u >= Fx) {
			pr <- pr * lambda / (i + 1)
			Fx <- Fx + pr
			i <- i + 1
		}
		return(i)
	}
	rpoisson <- function(n, lambda) {
		replicate(n, expr = rpois_aux(lambda), simplify = TRUE)
	}		
			\end{lstlisting}
		\end{column}
		
		\begin{column}{0.15\textwidth}
			\href{https://github.com/andresavassi/Trabalho-1---MCCD-II/blob/main/poisson_inversa.R}{\beamergotobutton{Código.R}}
			
			\vspace{0.5cm}

		\end{column}
		
	\end{columns}
\end{frame}

\begin{frame}{Resultados - Poisson Inversa}
	\centering
	\begin{overpic}[width=0.9\textwidth]{tbl_pois_inversa.png}
		% coloca o gráfico por cima
		\put(0,0){\includegraphics[width=0.6\textwidth]{graf_pois_inversa.png}}
	\end{overpic}
\end{frame}



\begin{frame}[fragile]{Amostrador Poisson via Aceitação/ Rejeição}
	\begin{columns}
		
		\begin{column}{0.85\textwidth}
			\begin{lstlisting}[language=R]
	set.seed(20252)
	rpois_ar <- function(lambda) {
		p <- 1 / (1 + lambda)
		f <- function(k) exp(-lambda) * lambda^k / factorial(k)
		g <- function(k) p * (1 - p)^k
		ks <- 0: (lambda*10 + 10)
		M <- max(f(ks) / g(ks))		
		repeat {
			X <- rgeom(1, prob = p)
			U <- runif(1)
			if (U <= f(X) / (M * g(X))) {
				return(X)
			}
		}
	}	
			\end{lstlisting}
		\end{column}
		
		\begin{column}{0.15\textwidth}
		\href{https://github.com/andresavassi/Trabalho-1---MCCD-II/blob/main/poisson_ar.R}{\beamergotobutton{Código.R}}
		
		\vspace{0.5cm}
		
		\end{column}
	
	\end{columns}
\end{frame}


	\begin{frame}{Gráficos - Poisson Aceitação e Rejeição}
	\begin{figure}[h]
		\centering
		\includegraphics[width=0.8\textwidth]{graf_poisson_ar.png}
	\end{figure}
\end{frame}



	\begin{frame}{Gerando amostras de distribuições - Binomial Negativa}
		Segue que a densidade da distribuição Binomial Negativa é descrita por:
		$$f(y; r,p) = \binom{y-1}{r-1} p^r(1-p)^{y-r}$$
		\ onde $r$ é o total de sucessos e $p$ é a probabilidade desses sucessos.
		Note que $Y \sim BN(r,p)$ possui esperança e variância a seguir:
		$$
		\mathbb{E}(Y) = \frac{r}{p}\ ;\ Var(Y) = \frac{r(1-p)}{p^2}
		$$
		
		
	\end{frame}
	
	
	\begin{frame}[fragile]{Amostrador BN via Transformação Inversa}
			\begin{columns}
			
			\begin{column}{0.85\textwidth}
				\begin{lstlisting}[language=R]
		r_param <- 4      
		p_param <- 0.5    
		n_amostra <- 1000 
		gerar_bn_inversa_aux <- function(r, p) {
			u <- runif(1) 				
			i <- 0
			pr <- p^r      
			Fx <- pr       
			while (u >= Fx) {
				pr <- pr * (i + r) / (i + 1) * (1 - p)
				Fx <- Fx + pr
				i <- i + 1
			}
			return(i)
		}
		gerar_bn_inversa <- function(n, r, p) {
			return(replicate(n, gerar_bn_inversa_aux(r, p)))
		}
				\end{lstlisting}
			\end{column}
			
			\begin{column}{0.15\textwidth}
				\href{https://github.com/andresavassi/Trabalho-1---MCCD-II/blob/main/bn_inversa.R}{\beamergotobutton{Código.R}}
				
				\vspace{0.5cm}
				
			\end{column}
			
		\end{columns}
		\end{frame}
		
			\begin{frame}{Gráficos - BN Transformação Inversa}
			\begin{figure}[h]
				\centering
				\includegraphics[width=0.8\textwidth]{graf_bn_inv.png}
			\end{figure}
		\end{frame}
		
		\begin{frame}[fragile]{Amostrador BN via Amostragem por Importância}
			\begin{columns}
				
				\begin{column}{0.85\textwidth}
					\begin{lstlisting}[language=R]
			r_param <- 4      
			p_param <- 0.5    
			n_amostra <- 1000 
			m_candidatos <- 20000 
			gerar_bn_sir <- function(n, r, p, m) {
				media_bn <- r * (1 - p) / p
				p_g <- 1 / (1 + media_bn)
				candidatos <- rgeom(m, prob = p_g)
				pesos_proposta <- dgeom(candidatos, prob = p_g)
				pesos_importancia <- pesos_alvo / pesos_proposta
				amostra_final <- sample(
				x = candidatos, 
				size = n, 
				replace = TRUE, 
				prob = pesos_importancia
				)
				return(amostra_final)
			}
					\end{lstlisting}
				\end{column}
				
				\begin{column}{0.15\textwidth}
					\href{https://github.com/andresavassi/Trabalho-1---MCCD-II/blob/main/bn_sir.R}{\beamergotobutton{Código.R}}
					
					\vspace{0.5cm}
					
				\end{column}
				
			\end{columns}
		\end{frame}
		
					\begin{frame}{Gráficos - BN Amostragem por Importância}
			\begin{figure}[h]
				\centering
				\includegraphics[width=0.8\textwidth]{graf_bn_sir.png}
			\end{figure}
		\end{frame}
		
				\begin{frame}[fragile]{Amostrador BN via Aceitação e Rejeição}
			\begin{columns}
				
				\begin{column}{0.85\textwidth}
					\begin{lstlisting}[language=R]
			r_param <- 4
			p_param <- 0.5
			n_amostra <- 1000 
			gerar_bn_rejeicao <- function(n, r, p) {
				media_bn <- r * (1 - p) / p
				p_g <- 1 / (1 + media_bn) 
				k_vals <- 0:100 
				razao_pq <- dnbinom(k_vals, size = r, prob = p) / dgeom(k_vals, prob = p_g)
				c <- max(razao_pq, na.rm = TRUE) * 1.01 
				amostra <- numeric(n)
				contador_aceitos <- 0
					\end{lstlisting}
				\end{column}
				
				\begin{column}{0.15\textwidth}
					\href{https://github.com/andresavassi/Trabalho-1---MCCD-II/blob/main/bn_ar.R}{\beamergotobutton{Código.R}}
					
					\vspace{0.5cm}
					
				\end{column}
				
			\end{columns}
		\end{frame}
		
		\begin{frame}[fragile]{Amostrador BN via Aceitação e Rejeição}
			\begin{lstlisting}
		while (contador_aceitos < n) {
			w <- rgeom(1, prob = p_g)
			u <- runif(1)
			p_w <- dnbinom(w, size = r, prob = p)
			q_w <- dgeom(w, prob = p_g)
			if (u < p_w / (c * q_w)) {
				contador_aceitos <- contador_aceitos + 1
				amostra[contador_aceitos] <- w
			}
		}
		return(amostra)
		}
			\end{lstlisting}
		\end{frame}


		\begin{frame}{Gráficos - BN Aceitação e Rejeição}
		\begin{figure}[h]
			\centering
			\includegraphics[width=0.8\textwidth]{graf_bn_ar.png}
		\end{figure}
	\end{frame}

	\begin{frame}{Gerando amostras de distribuições - Bell}
		Segue que a densidade da distribuição Bell é descrita por:
		$$
		f(y; \theta) = \frac{\theta e^{e^{\theta} + 1}}{y!} B_Y,\ y = 0, 1, \dots, n; \ \theta > 0
		$$
		
		Onde o termo $B_Y$ corresponde ao número de Bell, dados por:
		
		$$
		B_n= \frac{1}{e} \sum_{k = 0}^{\infty} \frac{k^n}{k!}, \text{iniciando com}\ B_0 = B_1 = 1
		$$
		
		A média e variância de $Y \sim Bell(\theta)$ é:
		$$
		\mathbb{E}(Y) = \theta e^{\theta}\ ;\ 
		Var(Y) = \theta(1+\theta)e^{\theta}
		$$
		
	\end{frame}
	
	\begin{frame}[fragile]{Amostrador Bell via Transformação Inversa}
		\begin{columns}
			
			\begin{column}{0.85\textwidth}
				\begin{lstlisting}[language=R]
					n <- 10000
					
					u <- runif(n)
					
					x_inv <- qnorm(u, mean = 0, sd = 1)
					
							
				\end{lstlisting}
			\end{column}
			
			\begin{column}{0.15\textwidth}
				\href{https://github.com/andresavassi/Trabalho-1---MCCD-II/blob/main/bell_inversa.R}{\beamergotobutton{Código.R}}
				
				\vspace{0.5cm}
				
			\end{column}
			
		\end{columns}
	\end{frame}
	
	\begin{frame}{Gráficos - Bell Inversa}
	\begin{figure}[h]
		\centering
		\includegraphics[width=0.8\textwidth]{graf_bell_inv.png}
	\end{figure}
\end{frame}

	\begin{frame}[fragile]{Amostrador Bell via Box Muller}
		\begin{columns}
			
			\begin{column}{0.85\textwidth}
				\begin{lstlisting}[language=R]
					n <- 10000
					
					u1 <- runif(n)
					u2 <- runif(n)
					
					z1 <- sqrt(-2 * log(u1)) * cos(2 * pi * u2)
					z2 <- sqrt(-2 * log(u1)) * sin(2 * pi * u2)
					
					x_bm <- c(z1, z2)
				\end{lstlisting}
			\end{column}
			
			\begin{column}{0.15\textwidth}
				\href{https://github.com/andresavassi/Trabalho-1---MCCD-II/blob/main/bell_bm.R}{\beamergotobutton{Código.R}}
				
				\vspace{0.5cm}
				
			\end{column}
			
		\end{columns}
	\end{frame}
	
	\begin{frame}{Gráficos - Bell Box Muller}
		\begin{figure}[h]
			\centering
			\includegraphics[width=0.8\textwidth]{graf_bell_bm.png}
		\end{figure}
	\end{frame}
	

	\section{Implementação}
	\begin{frame}[fragile]{Código}
		\begin{columns}
			
			\begin{column}{0.85\textwidth}
				\begin{lstlisting}[language=R]
					MC <- 1000
					nsize <- 100
					library(foreach)
					library(dplyr)
					library(tibble)
					library(doParallel)	
					set.seed(9999)
					df_teste <- data.frame(
					x1 = sample(c(0, 1), 10, replace = T, prob = c(0.7, 0.3)),
					x2 = rnorm(10),
					x3 = rnorm(10),
					x4 = sample(c(0, 1), 10, replace = T, prob = c(0.2, 0.8))
					)
					X_teste <- model.matrix(~., df_teste)
					results <- data.frame()	
				\end{lstlisting}
			\end{column}
			
			\begin{column}{0.15\textwidth}
				\href{https://github.com/andresavassi/Trabalho-1---MCCD-II/blob/main/mc_poisson.R}{\beamergotobutton{Código.R}}
				
				\vspace{0.5cm}
				
			\end{column}
			
		\end{columns}
	\end{frame}
	
	\begin{frame}[fragile]{Código}
				\begin{lstlisting}[language=R]
			run_MC <- function(r, nsize) {
				set.seed(r)
				n <- nsize
				betas <- c(1.2, 0.25, -0.08, 0.15, -0.12)
				
				df <- data.frame(
				x1 = sample(c(0, 1), n, replace = T, prob = c(0.7, 0.3)),
				x2 = rnorm(n),
				x3 = rnorm(n),
				x4 = sample(c(0, 1), n, replace = T, prob = c(0.2, 0.8))
				)
				X <- model.matrix(~., df)
				eta <- X %*% betas
				mu <- exp(eta)
			\end{lstlisting}
		\end{frame}
	


\end{document}
