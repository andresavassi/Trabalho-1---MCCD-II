\documentclass{beamer} % Define a classe do documento como Beamer

\usepackage{listings}
\usepackage{xcolor} % necessário para as cores no listings
\usepackage{listings}
\usepackage{xcolor}
\usepackage{graphicx}
\usepackage[percent]{overpic}
\usepackage{fancyvrb}

\lstset{
	language=R,
	basicstyle=\ttfamily\small,
	keywordstyle=\color{blue},
	commentstyle=\color{green!50!black},
	literate={\#}{{\texttt{\#}}}1,  % Processa o # corretamente
	% ... outras configurações
}

\usetheme{Madrid}   
\graphicspath{{figuras/}}  

% Informações do Título
\title{Simulated Annealing \\ (Recozimento Simulado)}
\author{}
\institute{UFMG}
\date{10/11/2025} % Data da apresentação

% CONFIGURAÇÃO PERSONALIZADA DO RODAPÉ
\setbeamertemplate{footline}{
	\leavevmode%
	\hbox{%
		% ESQUERDA: "grupo 2"
		\begin{beamercolorbox}[wd=.33\paperwidth,ht=2.25ex,dp=1ex,left]{author in head/foot}%
			\hspace{0.5cm}\textbf{Grupo II}%
		\end{beamercolorbox}%
		% CENTRO: "MCCD -2"
		\begin{beamercolorbox}[wd=.34\paperwidth,ht=2.25ex,dp=1ex,center]{title in head/foot}%
			\textbf{MCCD - II}%
		\end{beamercolorbox}%
		% DIREITA: Número da página (opcional)
		\begin{beamercolorbox}[wd=.33\paperwidth,ht=2.25ex,dp=1ex,right]{date in head/foot}%
			\textbf{10/11/2025}
			\insertpagenumber\hspace{0.5cm}%
	\end{beamercolorbox}}%
	\vskip0pt%
}

% Início do Documento
\begin{document}
	
	% Slide de Título
	\begin{frame}
		\titlepage % Gera o slide de título com as informações acima
	\end{frame}
	
	% Sumário (Opcional, mas recomendado)
	\begin{frame}{Sumário}
		\tableofcontents % Gera um sumário automático baseado nas seções
	\end{frame}
	\footnotesize
	
	
	\section{Simulação de Recozimento}
	\subsection{Introdução}
	\begin{frame}{Recozimento Simulado - Introdução}
	Também chamado de método meta-heurístico, este tem como intuito resolver problemas de otimização de grande complexidade. Este algoritmo foi introduzido por volta de 1980 por três pesquisadores, sendo eles: Kirkpatrick, Gelatt e Vecchi. O método traz soluções sub-ótimas, ou seja, eficientes e sem grande esforço computacional.
	
	Tem como objetivo encontrar resultados satisfatórios, ainda que não exatos, sendo que muitas vezes estes resultados são inviáveis e/ou difíceis de serem alcançados.
	
	Se enquadra na classe \textit{Markov chain Monte Carlo} (MCMC) e, por ter caráter estocástico alinhado à sua capacidade de escapar de mínimos locais, este é um ótimo candidato para resolução de problemas complexos de otimização.
	
	
	\end{frame}
	
	\subsection{Processo de Recozimento}
	
	\begin{frame}{Processo de Recozimento}
	O conceito foi introduzido da ideia de recozimento, onde um sólido é levado para um estado de baixa energia após aumentar sua temperatura.
	
	O processo consiste em dua estapas, sendo elas:
	
	\begin{itemize}
	\item "Derretimento" da sua estrutura ao levá-lo a uma temperatura muito elevada.
	\item Resfriamento esquematizado buscando atingir um estado sólido de energia mínima.
	\end{itemize}
	Uma vez em estado líquido as partículas do material exposto são distribuídas de forma aleatória.
	
	O estado mínimo de energia só é alcançado com uma temperatura inicial que seja suficientemente alta e um tempo de resfriamento que seja logo o suficiente.
	\end{frame}
	
	
	\subsection{Aplicabilidade}
	\begin{frame}{Aplicabilidade}
	 O algoritmo \textbf{Simulated Annealing} tem aplicabilidade em várias áreas e aqui serão expostas algumas delas:
	 \begin{itemize}
	 \item Engenharia de software
	 \item \textit{Machine Learning}
	 \item Teoria de filas
	 \item Processos de manufatura
	 \item logística e transporte com otimização
	 \end{itemize}
	\end{frame}
	
	
	\subsection{Teoria}
	\begin{frame}{Teoria}
	 
	\end{frame}
	
	\subsection{Implementação}
		
\begin{frame}[fragile]{Código implementado}
	\noindent
	{\color{blue} \footnotesize
		\begin{verbatim}
#bibliotecas
library(ggplot2); library(dplyr); library(tidyr); library(gridExtra)

#log-verossimilhança
loglik_weibull3 <- function(params, x) {
  b <- params[1]; g <- params[2]; c <- params[3]
  if (b <= 0 || g <= 0 || any(x <= c)) {
    return(-Inf)
  }
  z <- (x - c) / g
  ll <- length(x) * (log(b) - log(g)) + (b - 1) * sum(log(z)) - sum(z^b)
  return(ll)
}
    \end{verbatim}
	}
	\end{frame}
	
	\begin{verbatim}
#bibliotecas
library(ggplot2); library(dplyr); library(tidyr); library(gridExtra)

#log-verossimilhança
loglik_weibull3 <- function(params, x) {
  b <- params[1]; g <- params[2]; c <- params[3]
  if (b <= 0 || g <= 0 || any(x <= c)) {
    return(-Inf)
  }
  z <- (x - c) / g
  ll <- length(x) * (log(b) - log(g)) + (b - 1) * sum(log(z)) - sum(z^b)
  return(ll)
}
    \end{verbatim}
	}
	
	\end{document}